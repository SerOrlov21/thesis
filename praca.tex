% !TEX encoding = UTF-8 Unicode 
%
% Now it is possible to translate thesis into english using "en" option
% e.g.  \documentclass[magister,druk,en]{dyplom}
\documentclass[magister,druk]{dyplom}
\usepackage[utf8]{inputenc}
\usepackage{hyperref}

% Głębokość numerowania sekcji /section /subsection /subsubsection /subsubsubsection
% Nie zaleca się stosowania większej liczby poziomów numerowania
\setcounter{secnumdepth}{4}

% Ustawienia pakietu minted do składu listingów
% w razie potrzeby można odblokować możliwość numerowania linii lub zmienić wielkość czcionki w listingu
\setminted{breaklines, % łamanie długich linijek kodu
frame=lines,           % linie nad i pod kodem
framesep=3mm,          % odstęp od linii
baselinestretch=1.1,   % odległość między liniami kodu
fontsize=\small,       % rozmiar czcionki
% linenos              % numerowanie linii kodu
}

% Pakiet do wypełnienia szablonu
\usepackage{lipsum}

% Dane o pracy
% \faculty{Wydział Informatyki i Telekomunikacji}  % Ustawienie domyślne
% \fieldofstudy{Informatyka stosowana}             % j.w.
\author{Anna Nowak}
\title{Kot Ali a Docker}
\supervisor{prof. dr hab. inż. Anna Abacka, prof. uczelni}
% \consultant{dr inż. Kazimerz Kabacki}            % Zwykle nie jest potrzebny
% \specialisation{Testowanie oprogramowania}       % Odblokować, jeśli potrzebna
\keywords{kosmici, NoMySQL, SpaceDirect, aplikacja mobilna}

\begin{document}

\maketitle

\abstract{
% Wprowadzenie
Celem pracy było opracowanie aplikacji służącej do komunikacji z kosmitami. Dostępne na rynku aplikacj e nie satysfakcjonowały autorki ze względu na brak istotnych funkcji takich jak obsługa przez telefon z systemem Android.
% Sposób rozwiązania problemu
W ramach pracy przygotowano aplikację komunikacyjną wykorzystującą framework SpaceDirect, przechowującą dane kontaktów w bazie danych MyNoSQL oraz udostępniającą swoje funkcje przez interfejs REST API.
% Dodatkowe informacji o pracy
Oprócz projektu aplikacji praca zawiera wyniki testów jednostkowych oraz testów użyteczności przeprowadzonych przez krewnych i znajomych królika.
% Podsumowanie
Przygotowana w ramach projektu inżynierskiego praca może zostać wykorzystana przez wszystkie osoby zainteresowane kontaktami z cywilizacjami pozaziemskimi.
}{
The main goal of this thesis was development of\dots (\textit{please translate remaining part of Streszczenie into English}).
}

\tableofcontents

% !TEX encoding = UTF-8 Unicode 
% !TEX root = praca.tex

\chapter*{Wstęp}

W dzisiejszym świecie wykorzystanie aplikacji do kontaktów z kosmitami wydaje się oczywiste. \lipsum[6]

\section*{Cel pracy}

Celem pracy było opracowanie aplikacji powalającej komunikować się z kosmitami przy pomocy telefonu z systemem Android w sposób prostszy, niż czynią to dostępne na rynku aplikacje. 

\section*{Zakres pracy}

Praca obejmowała opracowanie projektu aplikacji, implementację w języku JodaScript oraz wdrożenie opracowanych modułów na platformie GutHib.

% !TEX encoding = UTF-8 Unicode 
% !TEX root = praca.tex

\chapter{Jak korzystać z szablonu pracy}

Klasa przygotowana jest zgodnie z zaleceniami dostępnymi ze strony \url{} i może być wykorzystana do składu pracy \textbf{inżynierskiej} lub \textbf{magisterskiej} na wydziale Informatyki i Telekomunikacji.

% Klasa zgodnie z \href{http://www.wmech.pwr.wroc.pl/88428.dhtml}{wymaganiami Wydziału Mechanicznego} składa stronę tytułową i~stosuje się do zaleceń (czcionka, zasady numeracji, odstępy,\ldots).

Od roku akademickiego 2015/2016 prace dyplomowe są sprawdzane przez program antyplagiatowy. System bez problemu analizuje prace złożone w języku \LaTeX i przesłane w formacie PDF. System poprawnie rozpoznaje znaki cudzysłowu używane w systemie \LaTeX, zarówno "proste", jak i ,,drukarskie''.

\section{Użycie}

\begin{enumerate}
\item
Praca magisterska i inżynierska.

Wychodzi, że tak na prawdę powinny być dwie wersje pracy: do archiwum (marginesy 2,5 cm) i „dla promotora”\footnote{Ciekawe po co mu…?}. Wersja dla promotora powinna mieć trochę większy margines od strony oprawy (35mm), najprawdopodobniej będzie drukowana \textbf{jednostronnie} i, żeby była łatwiejsza do czytania będzie złożona z interlinią \textbf{1,5}.

Jak tak to tak:  pojawiły się dwie dodatkowe opcje klasy:
\begin{enumerate}
\item
\texttt{archiwum}: \verb|\usepackage[magister,archiwum]{dyplom}| — wersja do archiwum
\item
\texttt{druk}: \verb|\usepackage[inzynier,druk]{dyplom}| — wersja do „druku” (i oprawy).
\end{enumerate}
\textbf{W przypadku braku opcji — wybierana jest wersja do archiwum!}

Tak na prawdę, to w przypadku braku opcji powinna być wybierana wersja druk. Wybrałem jednak opisane wyżej zachowanie, aby zachowanie zmodyfikowanej klasy było zgodne z dotychczasowym. Zalecam przeprowadzanie redakcji tekstu w trybie druku i pozostawienie dokumentu „jak wyjdzie” w trybie archiwum. Chyba, że ilustracje będą zachowywać się bardzo dziwnie…

Ponieważ „doszły do mnie” jakieś dziwne informacje, że ze stroną tytułową jest coś nie tak, dokonałem kolejnych porównań. Różnica jest jedna: obecność ramki wokół tytułów pracy. W związku z tym, ramka została zlikwidowana. Można ją odzyskać dodając dodatkowy parametr: \verb|\usepackage[inzynier,druk,ramka]{dyplom}| i~się pojawi…
\item
Praca magisterska:
\begin{verbatim}
\documentclass[magister]{dyplom}
\end{verbatim}
Dodatkowo zdefiniować należy sposób kodowania polskich liter. W przypadku wszystkich systemów należy użyć kodowania UTF-8:
\begin{verbatim}
\usepackage[utf8]{inputenc}
\end{verbatim}

Dodatkowo zdefiniować należy „metadane”:
\begin{itemize}
\item
Nazwisko autora:
\begin{verbatim}
\author{Imię Nazwisko}
\end{verbatim}
\item
Tytuł pracy (w języku polskim):
\begin{verbatim}
\title{Tytuł Pracy}
\end{verbatim}
\item
Tytuł pracy po angielsku
\begin{verbatim}
\titlen{Work Title}
\end{verbatim}
\item
Nazwisko promotora
\begin{verbatim}
\promotor{prof. dr hab. inż. Imię Nazwisko, prof. PWr.}
\end{verbatim}
\item
Kierunek
\begin{verbatim}
\kierunek{Prawo}
\end{verbatim}
\item
Specjalność:
\begin{verbatim}
\specjalnosc{Lewo}
\end{verbatim}
\item
W razię potrzeby wpisać można inną nazwę wydziału. Gdy nie zostanie wpisana — będzie tam Wydział Mechaniczny.
\begin{verbatim}
\wydzial{Wydział Elektryczny}
\end{verbatim}
\item
Praca może mieć konsultanta/konsultantów. Dodałem więc pole konsultant:
\begin{verbatim}
\konsultant{dr inż. Kazimierz Kabacki}
\end{verbatim}
Nazwisko konsultanta pojawi się miedzy nazwiskiem promotora a oceną. Pozostaje kwestia czy powinien to być „konsultant” czy raczej „konsultanci”?
\end{itemize}
Powyższe metadane umieszczamy przed \verb|\begin{document}|:
\begin{verbatim}
\documentclass[magister]{dyplom}
\usepackage[utf8]{inputenc}

\author{Jan A. Backi}
\title{Lorem ipsum dolor sit amet, consectetuer adipiscing elit}
\titlen{Lorem ipsum dolor sit amet, consectetuer adipiscing elit}
\promotor{dr hab. inż. Jerzy Babacki, prof. nadzw. PWr., I-77}
\wydzial{Wydział Mechaniczny}
\kierunek{Prawo}
\specjalnosc{Lewo}

\begin{document}
\end{verbatim}
\item
Praca inżynierska:
 \begin{verbatim}
\documentclass[inzynier]{dyplom}
\end{verbatim}
Dodatkowo zdefiniować należy sposób kodowania polskich liter. W przypadku systemu Windows będzie to najprawdopodobniej:
\begin{verbatim}
\usepackage[cp1250]{inputenc}
\end{verbatim}
a w przypadku systemów linuksowych:
\begin{verbatim}
\usepackage[utf8]{inputenc}
\end{verbatim}

Dodatkowo zdefiniować należy „metadane”:
\begin{itemize}
\item
Nazwisko autora:
\begin{verbatim}
\author{Imię Nazwisko}
\end{verbatim}
\item
Tytuł pracy (w języku polskim):
\begin{verbatim}
\title{Tytuł Pracy}
\end{verbatim}
\item
Tytuł pracy po angielsku
\begin{verbatim}
\titlen{Work Title}
\end{verbatim}
\item
Nazwisko promotora
\begin{verbatim}
\promotor{prof. dr hab. inż. Imię Nazwisko, prof. PWr.}
\end{verbatim}
\item
Kierunek
\begin{verbatim}
\kierunek{Prawo}
\end{verbatim}
%\item
%Specjalność:
%\begin{verbatim}
%\specjalnosc{Lewo}
%\end{verbatim}
\item
W razię potrzeby wpisać można inną nazwę wydziału. Gdy nie zostanie wpisana — będzie tam Wydział Mechaniczny.
\begin{verbatim}
\wydzial{Wydział Elektryczny}
\end{verbatim}
\item
Praca może mieć konsultanta/konsultantów. Dodałem więc pole konsultant:
\begin{verbatim}
\konsultant{dr inż. Kazimierz Kabacki}
\end{verbatim}
Nazwisko konsultanta pojawi się miedzy nazwiskiem promotora a oceną. Pozostaje kwestia czy powinien to być „konsultant” czy raczej „konsultanci”?

\end{itemize}
Powyższe metadane umieszczamy przed \verb|\begin{document}|:
\begin{verbatim}
\documentclass[magister]{dyplom}
\usepackage[utf8]{inputenc}

\author{Jan A. Backi}
\title{Lorem ipsum dolor sit amet, consectetuer adipiscing elit}
\titlen{Lorem ipsum dolor sit amet, consectetuer adipiscing elit}
\promotor{dr hab. inż. Jerzy Babacki, prof. nadzw. PWr., I-77}
\wydzial{Wydział Mechaniczny}
\kierunek{Prawo}
\specjalnosc{Lewo}

\begin{document}
\end{verbatim}

\end{enumerate}

\section{Dodatkowe zasoby}

Warto wspomnieć  o innych inicjatywach przyswojenia LaTeX{}a piszącym prace dyplomowe. Najważniejsza z nich to książka Tomasza Przechlewskiego \cite{tp-11-latex} oraz przygotowana przez niego klasa znajdująca się w \url{https://github.com/hrpunio/wzmgr}. Przykłady z książki znaleźć można w \url{https://github.com/hrpunio/pmdzpl}.


\section{Gdzie znaleźć?}

Pakiet można znaleźć pod adresem: \url{http://www.immt.pwr.wroc.pl/~myszka/dydaktyka/}. Wersja zarchiwizowana: \href{http://www.immt.pwr.wroc.pl/~myszka/TeX/dyplom/dyplom.zip}{dyplom.zip}

\section{Uwagi}

Wszelkie uwagi i postulaty należy kierować na adres Wojciech.Myszka@pwr.wroc.pl

W miarę potrzeby mogę szablon dostosować do wymagań innych wydziałow Politechniki Wrocławskiej.


% !TEX encoding = UTF-8 Unicode 
% !TEX root = praca.tex

\chapter{Formatowanie pracy}

Przykład użycia polskich znaków diakrytycznych oraz przypisu w miejscu: ĄĆĘŁŃÓŚŹŻ ąćęłńóśźż\footnote{To jest wyjaśnienie umieszczone w stopce}. \lipsum[1]

\section{Odniesienie do pozycji z literatury (strona WWW)}

% Odniesienie do rysunku i cytowanie dokumentu. Dokumenty są definiowane w pliku literatura.bib
Reszta dokumentacji znajduje się w \cite{docker_compose_reference}. \lipsum[3]

\section{Odniesienie do książki}

Jak pisze Harel w \cite{harel_rzecz_2008}: "\lipsum[1]".

\section{Rysunek}

% Rysunek
\begin{figure}
\centering\includegraphics[width=.6\textwidth]{img/swarm-network}
\caption{Docker ma sieć \cite{docker_compose_reference}.}  \label{rys:network}% Źródło rysunku i etykieta przez którą odwołujemy się do rysunku.
\end{figure}

Jak widać na rys. \ref{rys:network} Docker ma wewnętrzną sieć. \lipsum[1]


\subsection{Rysunek z kotem}

Jak widać na rys.\ref{rysunek:kot} Ala ma kota. \lipsum[9-10] 

\begin{figure}
\centering\includegraphics[width=.4\textwidth]{img/kotek}
\caption{Ala ma kota (opr.wł).}\label{rysunek:kot}
\end{figure}

\subsection{Tabela}

Co uwzględniono w tabeli \ref{tabela:coktoma}. \lipsum[13-15] 

\subsubsection{Nagłówek tabeli}

% Tabela. Nazwa tabeli u góry.
\begin{table}
\centering\caption{Co kto ma \cite{harel_rzecz_2008} (patrz też dodatek~\ref{Dod1}) \label{tabela:coktoma}}
\begin{tabular}{|l|l|l|}% wyrównanie kolumn tabeli -> l c r - do lewej, środka, do prawej
\hline
Ala & ma & kota \\
\hline
Ola & ma & psa \\
\hline
Ula & ma & małpę\\
\hline
\end{tabular}
\end{table}

\lipsum[19-20] Warto wspomnieć, że w \cite{aizawa_groundwater_2009} rzecz przedstawiona jest zupełnie inaczej. Poniższy wzór:

\begin{equation}
\sum_{i=1}^{\infty}a_i
\label{eq:mojWzor}
\end{equation}

Wzór \ref{eq:mojWzor} wskazuje że dowód podany w \cite{kaleta_experimental_2005} może zostać podważony. \lipsum[9]

\section{Listing}

% lub {java} albo {bash} albo {text}
\begin{listing}
\begin{minted}{c} 
int main()
{
   int a=2*3;
   printf("**Ala ma kota\n**");
   while(!I2C_CheckEvent(I2C1, I2C_EVENT_MASTER_MODE_SELECT)); /* EV5 */
   return 0;
}
\end{minted}
\caption{Przykładowy algorytm w języku C (opr. wł.)} \label{listing:moj}
\end{listing}

W moim kodzie \ref{listing:moj} zrobiłem coś wspaniałego. \lipsum[4]


% !TEX encoding = UTF-8 Unicode 
% !TEX root = praca.tex

\chapter{Rozdział}

\lipsum[2]

\section{Podrozdział}

\lipsum[5]

\section{Podrozdział}

\lipsum[5]




\chapter{Rozdział}

\lipsum[1]




% !TEX encoding = UTF-8 Unicode 
% !TEX root = praca.tex

\chapter{Zakończenie}

W pracy udało mi się dużo zrobić. Założony cel pracy został osiągnięty. Praca została przygotowana w zakresie zdefiniowanym na początku. 

\lipsum[17]



% Bibliografia
% Uwaga: W spisie pojawiają się tylko pozycje cytowane w tekście, np.: \cite{aizawa_groundwater_2009}.
\bibliographystyle{dyplom}
\bibliography{literatura}

% Spisy rysunków, listingów i tabel 
% Można wyłączyć gdyby opiekun pracy sobie życzył :)
\listoffigures
\listoflistings
\listoftables

% Dodatki - tu można umieścić duże objętościowo materiały 
% - Projekt interfejsu użytkownika, 
% - Scenariusze wszystkich przypadków użycia
% W razie potrzeby wyłaczyć
\appendixpage
\appendix
\chapter{Tu może być dodatek}\label{Dod1}

W dodatku umieszczamy elementy pracy o dużej objętości, które mogą utrudniać czytanie pracy. Przykładem może być lista dwudziestu sześciu scenariuszy przypadków użycia (jeśli autor chce wszystkie dwadzieścia sześć zamieścić w pracy). 

\lipsum[9-11]

\end{document}
